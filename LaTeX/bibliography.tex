\begin{thebibliography}{ABCDEF}

\renewcommand{\refname}{\normalsize Literaturverzeichnis}

%\bibitem[Abe01]{abend:01}
%Michael Abend. "'Robotik und Sensorik. Darstellungsschwerpunkt: Selbständige Entwicklung "`unscharfer"' Algorithmen zur räumlichen Orientierung (unter Verwendung des LEGO-Mindstorms-Systems)", \emph{Schriftliche Prüfungsarbeit zur zweiten Staatsprüfung für das Amt des Studienrats}, Berlin, 2001

\bibitem[Abt15]{abts:15}
Dietmar Abts. \emph{Grundkurs JAVA. Von Grundlagen bis zu Datenbank- und Netzwerkanwendungen}, 8., überarbeitete und erweiterte Auflage, Springer Vieweg, Wiesbaden, 2015

\bibitem[Aeb11]{aebli:11}
Hans Aebli. \emph{Zwölf Grundformen des Lehrens. Eine Allgemeine Didaktik auf psychologischer Grundlage. Medien und Inhalte didaktischer Kommunikation, der Lernzyklus}, 14. Auflage, Klett-Cotta, Stuttgart, 2011

\bibitem[Bar03]{barnes:03}
David J. Barnes, Michael Kölling. \emph{Objektorientierte Programmierung mit Java. Eine praxisnahe Einführung mit BlueJ}, Übersetzt von Axel Schmolitzky, Pearson Studium, München, 2003

\bibitem[Bre94]{breier:94}
Norbert Breier. \emph{Informatische Bildung als Teil der Allgemeinbildung}, LOG IN 14, H. 5/6., 1994
%\pagebreak 

\bibitem[Bzl16]{barzel:16}
Bärbel Barzel, Lars Holzäpfel, TImo Leuters, Christine Streit. \emph{Scriptor Praxis: Mathematik unterrichten: Planen, durchführen, reflektieren}, 4. Auflage, Cornelsen Schulverlage GmbK, Berlin, 2016

%\bibitem[Ber10]{berns:10}
%Karsten Berns, Daniel Schmidt. \emph{Programmierung mit LEGO MINDSTORMS NXT. Robotersysteme, Entwurfsmethodik, Algorithmen}, Springer Heidelberg Dordrecht London New York, 2010

%\bibitem[Bow12]{bowes:12}
%David Bowes. \url{http://homepages.herts.ac.uk/~comqdhb/lego/bluej.php}, Abgerufen am 07.02.2016, Herfortshire, 2012, Lejos NXJ extension for BlueJ

%\bibitem[BricxCC]{bricxcc}
%o.V. URL: \url{http://bricxcc.sourceforge.net/}, Abgerufen am 02.02.2016

\bibitem[Ehm09]{ehmann:09}
Matthias Ehmann et al. \emph{Duden Informatik - Sekundarstufe I / 9./10. Schuljahr - Objektorientierte Programmierung mit BlueJ}, Duden Schulbuchverlag Berlin Mannheim, 2009

\bibitem[GyOhm16]{ohmoor:16}
Fachschaft Informatik. \emph{Schulinternes Curriculum Informatik. Sekundarstufe II Wahlbereich und Profile}, Hamburg, Stand: 14.03.2016

%\bibitem[Her12]{hertzberg:12}
%Joachim Hertzberg, Kai Lingemann, Andreas Nüchter. \emph{Mobile Roboter. Eine Einführung aus Sicht der Informatik}, Springer-Verlag Berlin Heidelberg, 2012

\bibitem[HH09]{oberstufe:09}
Behörde für Schule und Berufsbildung Hamburg (Hrsg.). \emph{Informatik -- Bildungsplan  Gymnasiale Oberstufe}, Hamburg, 2009

\bibitem[HH11]{gymsek1:11}
Behörde für Schule und Berufsbildung Hamburg (Hrsg.). \emph{Informatik Wahlfplichtfach -- Bildungsplan Gymnasium Sekundarstufe I}, Hamburg, 2011

\bibitem[HH14]{stsmittel:14}
Behörde für Schule und Berufsbildung Hamburg (Hrsg.). \emph{Informatik Wahlpflichtfach -- Bildungsplan Stadtteilschule Jahrgangsstufen 7 -- 11}, Hamburg, 2014

\bibitem[Hub07]{hubwieser:07}
Peter Hubwieser. \emph{Didaktik der Informatik}, 3. Auflage, Springer-Verlag Berlin Heidelberg, 2007

\bibitem[Hum02]{humbert:02}
Ludger Humbert, Sigrid Schubert. \emph{Fachliche Orientierung des Informatikunterrichts in der Sekundarstufe II}, Didaktik der Informatik Universität Dortmund, Report Nr. 77, Februar 2002

\bibitem[Koe93]{koerber:93}
Bernhard Koerber, Ingo-Rüdiger Peters. \emph{Informatikunterricht und informationstechnische Grundbildung –
ausgrenzen, abgrenzen oder integrieren?}, Troitzsch, S. 108--115, 1993 

\bibitem[Mod11]{modrow:11}
Eckart Modrow. "{Visuelle Programmierung -- oder: Was lernt man aus Syntaxfehlern?}", In: Marco Thomas (Hrsg.): \emph{Informatik in Bildung und Beruf. 14. GI-Fachtagung "`Informatik und Schule – INFOS 2011"'}, S. 27--36, 2011

%\bibitem[Lego]{lego}
%o.V. URL: \url{http://www.lego.com/en-us/mindstorms/history}, Abgerufen am 06.11.2015, LEGO, 2015
%
%\bibitem[leJOS]{lejos}
%o.V. URL: \url{http://www.lejos.org/nxj.php}, Abgerufen am 14.12.2015, leJOS Java for Lego Mindstorms, 2015

%\bibitem[Lil08]{lilienthal:08}
%Carloa Lilienthal. "Komplexität von Softwarearchitekturen -- Stile und Strategien --", \emph{Dissertation im Fachbereich Informatik der Universität Hamburg}, Hamburg, 2008


%\bibitem[Sch04]{schreiber:04}
%Rafael Schreiber. "Der Einsatz von LEGO-Mindstorms im Informatikunterricht der 11. Klasse der Leonard-Bernstein-Oberschule. Sicherung und Transfer grundlegender algorithmischer Strukturen in NQC.", \emph{Schriftliche Prüfungsarbeit im Rahmen der zweiten Staatsprüfung für das Amt des Studienrats}, Berlin, 2004

\bibitem[Schwa07]{schwarzer:07}
Christine Schwarzer, Petra Buchwald. "{Umlernen und Dazulernen}.", In: Michael Göhlich, Christoph Wulf, Jörg Zirfas (Hrsg.): \emph{Pädagogische Theorien des Lernens}, Beltz, Weinheim und Basel,  S. 213--221, 2007


%\bibitem[RWTH]{rwth}
%o.V. URL:{http://schuelerlabor.informatik.rwth-aachen.de/simulator}, Abgerufen am 02.02.2016, Simulator für LEGO Mindstorms NXT Roboter

%\bibitem[Sto01]{stolt:01}
%Matthias Stolt. "Roboter im Informatikunterricht", 2001

\bibitem[Ull12]{ullenboom:12}
Christian Ullenboom. \emph{Java ist auch eine Insel -- Das umfassende Handbuch}, 10. Auflage, Galileo Press, Bonn, 2012

%\bibitem[Wag05]{wagner:05}
%Oliver Wagner. "LEGO Roboter im Informatikunterricht. Eine Untersuchung zum Einsatz des LEGO-Mindstorms-Systems zur Steigerung des Kooperationsvermögens im Informatikunterricht eines Grundkurses (12. Jahrgang, 2. Lernjahr) der Otto-Nagel-Oberschule (Gymnasium)", \emph{Schriftliche Prüfungsarbeit im Rahmen der zweiten Staatsprüfung für das Amt des Studienrats}, Berlin, 2005

%\bibitem[Zül90]{züllighoven:90}
%Reinhard Budde, Heinz Züllighoven. \emph{Software-Werkzeuge in einer Programmierwerkstatt. Ansätze eines hermeneutisch fundierten Werkzeug- und Maschinenbegriffs}, Oldenbourg, München [u.a.], 1990
\end{thebibliography}